\section{Inductive Learning and Decision Trees}
\textbf{Date:} \underline{Sep 2, 2025}

\subsection{Inductive Learning}

Inductive learning is the process of learning a general rule or hypothesis from specific observed examples.  
Given a training dataset:
\[
S = \{(x_1, y_1), (x_2, y_2), \dots, (x_n, y_n)\}
\]
we aim to find a hypothesis $h$ such that $h(x) \approx y$ for unseen examples.

\subsubsection*{Key Ideas}
\begin{itemize}
    \item Training data provides labeled examples of inputs and outputs.
    \item The goal is to infer a hypothesis $h$ consistent with as many training examples as possible.
    \item If multiple hypotheses are consistent, we aim to choose one that generalizes well.
\end{itemize}

\subsection{Version Space}

\begin{definition}[Version Space]
The \textbf{version space} is the set of all hypotheses in the hypothesis space $H$ that are consistent with the observed training examples:
\[
VS = \{ h \in H \;|\; \forall (x_i, y_i) \in S,\ h(x_i) = y_i \}.
\]
\end{definition}

\subsubsection*{Using Version Space for Learning}
\begin{itemize}
    \item Start with the set of all hypotheses.
    \item Remove any hypothesis inconsistent with any training example.
    \item The remaining hypotheses form the version space.
\end{itemize}

\subsection{List-Then-Eliminate Algorithm}

\begin{algobox}
\textbf{Algorithm:} \emph{List-Then-Eliminate}
\begin{enumerate}
    \item Initialize $VS \leftarrow H$ (all hypotheses).
    \item For each training example $(x_i, y_i)$:
    \begin{itemize}
        \item Remove all $h \in VS$ such that $h(x_i) \neq y_i$.
    \end{itemize}
    \item Return the remaining hypotheses in $VS$.
\end{enumerate}
\end{algobox}

\textbf{Takeaway:}  
Tracking the entire version space can be expensive in both time and memory.  
Instead, we often directly construct a consistent hypothesis, e.g., using decision trees.

\subsection{Decision Trees}

\begin{definition}[Decision Tree]
A decision tree represents a function $h: \mathcal{X} \to \mathcal{Y}$ as a tree structure:
\begin{itemize}
    \item \textbf{Internal nodes:} Test a single feature (e.g., "Color").
    \item \textbf{Branches:} Possible values of that feature (e.g., "Red" or "Green").
    \item \textbf{Leaf nodes:} Assign a label based on the path from root to leaf.
\end{itemize}
\end{definition}

\subsubsection*{Using a Decision Tree}
To classify a new example:
\begin{enumerate}
    \item Start at the root.
    \item At each internal node, test the corresponding feature.
    \item Follow the branch matching the feature value.
    \item Stop at a leaf and return its label.
\end{enumerate}

\subsection{Top-Down Induction of Decision Trees (IDT)}

\begin{algobox}
\textbf{Algorithm:} \emph{IDT(S, Features)}
\begin{enumerate}
    \item If all examples in $S$ have the same label, return a leaf node with that label.
    \item If no features remain, return a leaf node with the majority label in $S$.
    \item Otherwise:
    \begin{itemize}
        \item Choose the best feature $A$ to split on.
        \item Partition $S$ into subsets $\{S_v\}$ by the values of $A$.
        \item For each value $v$ of $A$:
        \begin{itemize}
            \item Recursively call \emph{IDT($S_v$, Features $\setminus \{A\}$)}.
        \end{itemize}
    \end{itemize}
\end{enumerate}
\end{algobox}

\subsection{Choosing the Best Split}

\textbf{Error-Based Split Criterion:}  
To choose the best attribute $A$ to split on:
\[
\text{Err}(S) = \min(\#\text{positive},\ \#\text{negative})
\]
\[
\text{Err}(S \mid A) = \sum_{v} \text{Err}(S_v)
\]
Select $A$ that maximizes:
\[
\Delta \text{Err} = \text{Err}(S) - \text{Err}(S \mid A).
\]

\subsection{Properties of Decision Trees}

\begin{itemize}
    \item Easy to interpret and visualize.
    \item Can represent complex decision boundaries.
    \item Handles both categorical and numerical features.
    \item Prone to overfitting if the tree grows too deep.
    \item Typically combined with pruning techniques for better generalization.
    \item Basis for more advanced models like Random Forests and Gradient Boosted Trees.
\end{itemize}