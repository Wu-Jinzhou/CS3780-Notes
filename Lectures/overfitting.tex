\section{Prediction and Overfitting}
\textbf{Date:} \underline{Sep 4, 2025}

\subsection*{Learning as Prediction}

\textbf{Goal:}  
Given a training dataset $S = \{(x_1, y_1), \dots, (x_n, y_n)\}$ drawn \textbf{i.i.d.} from an unknown distribution $\mathcal{D}$,  
learn a hypothesis $h$ such that $h(x) \approx y$ for unseen data.

\subsubsection*{World as a Distribution}
Features (e.g., Farm, Color, Size, Firmness) and labels (e.g., Tasty) are random variables. The underlying distribution $\mathcal{D}$ defines:
    \begin{itemize}
        \item \textbf{Joint Distribution:} $P(X, Y)$
        \item \textbf{Marginal Distribution:} $P(X)$
        \item \textbf{\subsection Distribution:} $P(Y|X)$
    \end{itemize}

\subsection*{Sample vs. Prediction Error}

\begin{definition}[Sample (Empirical) Error]
\[
\hat{L}_S(h) = \frac{1}{|S|} \sum_{(x_i,y_i)\in S} \mathbf{1}\big[h(x_i) \neq y_i\big]
\]
\end{definition}

\begin{definition}[Prediction (True) Error]
\[
L_{\mathcal{D}}(h) = \mathbb{P}_{(x,y)\sim\mathcal{D}}[h(x) \neq y]
\]
\end{definition}

Goal: Minimize \emph{true prediction error}, not just training error.

\subsection*{Overfitting}

Overfitting occurs when a hypothesis $h$ achieves \textbf{low training error} but \textbf{high test error} because it learns noise and idiosyncrasies of the training set instead of general patterns.

Example:
Take an i.i.d. training set $S = \{(x_1, f(x_1)), \dots\}$ and return $h_s$ such that 
$$h_s(x) = \begin{cases} f(x_i) & \text{if } x = x_i \text{ for some }i \\ \text{flip a coin} & \text{otherwise} \end{cases}$$
- $h_s$ has zero training error but predicts randomly on unseen data.

\subsubsection*{Key Characteristics}
\begin{itemize}
    \item Fits the training set too closely, including random noise.
    \item Poor generalization to unseen data.
    \item Common when the hypothesis space is very flexible (e.g., deep trees, high-degree polynomials).
\end{itemize}

\subsection*{Overfitting in Decision Trees}
\begin{itemize}
    \item Fully grown decision trees can perfectly memorize the training set.
    \item This leads to zero empirical error but poor generalization.
    \item Needs mechanisms like early stopping or pruning to avoid overfitting.
\end{itemize}

\subsection*{Mitigating Overfitting in Decision Trees}

\subsubsection*{Strategies}
\begin{itemize}
    \item \textbf{Limit Model Complexity:}  
          Restrict tree depth or number of nodes.
    \item \textbf{Early Stopping:}  
          Stop splitting when:
          \begin{itemize}
              \item Error reduction after splitting is small.
              \item Too few examples remain in a node.
          \end{itemize}
    \item \textbf{Pruning:}  
          Grow the full tree, then prune back:
          \begin{itemize}
              \item Replace a subtree with a leaf if it does not significantly increase prediction error.
              \item E.g., reduced-error pruning.
          \end{itemize}
\end{itemize}

\subsection*{Inductive Bias}

\begin{definition}[Inductive Bias]
An \textbf{inductive bias} is a set of assumptions a learning algorithm uses to predict unseen data.  
Without bias, learning from finite samples would be impossible.
\end{definition}

\subsubsection*{Inductive Bias in Decision Trees}
- Standard IDT assumes:
    \begin{itemize}
        \item The simplest consistent tree is preferred.
        \item Features are chosen based on information gain or error reduction.
    \end{itemize}
- If tree depth is restricted, bias increases but variance decreases.

\subsection*{Bias-Variance Tradeoff}

\begin{definition}[Prediction Error Decomposition]
\[
\text{Expected Error} = \underbrace{\text{Bias}^2}_{\text{error from assumptions}} + \underbrace{\text{Variance}}_{\text{error from data fluctuations}} + \underbrace{\sigma^2}_{\text{irreducible noise}}
\]
\end{definition}

\begin{itemize}
    \item \textbf{High bias:} Model too simple $\rightarrow$ underfits.
    \item \textbf{High variance:} Model too complex $\rightarrow$ overfits.
    \item Goal: Balance bias and variance for best generalization.
\end{itemize}
